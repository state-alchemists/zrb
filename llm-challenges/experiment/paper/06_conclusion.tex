\section{Conclusion}
\label{sec:conclusion}

We presented a comprehensive evaluation of 13 state-of-the-art Large Language Models across five representative software engineering tasks. Our multi-task benchmark addresses the gap in holistic SE evaluation, measuring both output quality and completion efficiency through automated verification.

\subsection{Summary of Contributions}

Our work makes four key contributions:

\begin{enumerate}
    \item \textbf{Comprehensive Benchmark:} We evaluated 13 models on bug fixing, feature development, refactoring, technical writing, and research synthesis---the broadest SE-focused comparison to date.
    
    \item \textbf{Efficiency Insights:} We revealed that completion time varies by 8$\times$ among top performers, with no correlation between time and quality. GPT-5.1 achieved perfect scores at 44s average, while GLM-4.7 required 355s for equivalent results.
    
    \item \textbf{Tool Usage Analysis:} We found that tool invocation frequency does not correlate with success ($r=0.077$, $p=0.575$). This challenges assumptions about ``thinking longer'' and suggests optimization opportunities in agent frameworks.
    
    \item \textbf{Practical Guidance:} Our findings provide evidence-based recommendations for model selection based on task type, speed requirements, and budget constraints.
\end{enumerate}

\subsection{Key Findings}

\begin{itemize}
    \item Four models (GPT-5.1, Gemini-3 Pro, Deepseek-Chat, GLM-4.7) achieved perfect 10/10 scores, demonstrating that current LLMs can handle diverse SE tasks with high proficiency.
    
    \item Coding tasks approached saturation with 100\% success across all models, suggesting the need for more challenging benchmarks in this domain.
    
    \item Research synthesis remained challenging (90.9\% success), with citation management being a particular weakness.
    
    \item OpenAI models demonstrated superior efficiency, completing tasks 3--8$\times$ faster than competitors without quality degradation.
\end{itemize}

\subsection{Impact}

This work enables practitioners to make informed LLM selection decisions based on:
\begin{itemize}
    \item Task-specific performance data
    \item Time-efficiency tradeoffs
    \item Cost-performance analysis
    \item Tool usage patterns
\end{itemize}

Researchers can build upon our benchmark to evaluate new models, test intervention strategies, and track capability evolution over time.

\subsection{Data Availability}

All experimental data, verification scripts, analysis code, and paper materials are available at:\\
\url{https://github.com/[anonymous]/llm-challenge-experiment}

\subsection{Future Directions}

As LLM capabilities continue to evolve, we anticipate:
\begin{itemize}
    \item Increasing differentiation in specialized tasks
    \item Greater emphasis on efficiency metrics alongside accuracy
    \item Standardization of multi-task SE benchmarks
    \item Integration of human-AI collaborative evaluation
\end{itemize}

We encourage the community to extend this benchmark, validate our findings, and contribute to the development of rigorous evaluation standards for LLMs in software engineering.

\vspace{0.5cm}
\noindent\textbf{Acknowledgments:} We thank the anonymous reviewers for their valuable feedback. This research was supported by [anonymous funding sources].
